% arara: makeindex

% Template for IEEE papers
%% bare_conf.tex
%% V1.4b
%% 2015/08/26
%% by Michael Shell
%% See:
%% http://www.michaelshell.org/
%% for current contact information.
%%
%% This is a skeleton file demonstrating the use of IEEEtran.cls
%% (requires IEEEtran.cls version 1.8b or later) with an IEEE
%% conference paper.
%%
%% Support sites:
%% http://www.michaelshell.org/tex/ieeetran/
%% http://www.ctan.org/pkg/ieeetran
%% and
%% http://www.ieee.org/

%%*************************************************************************
%% Legal Notice:
%% This code is offered as-is without any warranty either expressed or
%% implied; without even the implied warranty of MERCHANTABILITY or
%% FITNESS FOR A PARTICULAR PURPOSE!
%% User assumes all risk.
%% In no event shall the IEEE or any contributor to this code be liable for
%% any damages or losses, including, but not limited to, incidental,
%% consequential, or any other damages, resulting from the use or misuse
%% of any information contained here.
%%
%% All comments are the opinions of their respective authors and are not
%% necessarily endorsed by the IEEE.
%%
%% This work is distributed under the LaTeX Project Public License (LPPL)
%% ( http://www.latex-project.org/ ) version 1.3, and may be freely used,
%% distributed and modified. A copy of the LPPL, version 1.3, is included
%% in the base LaTeX documentation of all distributions of LaTeX released
%% 2003/12/01 or later.
%% Retain all contribution notices and credits.
%% ** Modified files should be clearly indicated as such, including  **
%% ** renaming them and changing author support contact information. **
%%*************************************************************************


% *** Authors should verify (and, if needed, correct) their LaTeX system  ***
% *** with the testflow diagnostic prior to trusting their LaTeX platform ***
% *** with production work. The IEEE's font choices and paper sizes can   ***
% *** trigger bugs that do not appear when using other class files.       ***                          ***
% The testflow support page is at:
% http://www.michaelshell.org/tex/testflow/

\documentclass{book}
% \def\nolinkurl{\url}
% \usepackage{nohyperref}
% \def\UrlBreaks{\do\/\do-}
\usepackage[quiet]{fontspec}
\usepackage[table,xcdraw,dvipsnames]{xcolor} % Used by spritegrid and others.
\usepackage[obeyspaces,spaces]{url}
\usepackage{longtable}
\usepackage{arydshln}
\usepackage{booktabs}
\usepackage{afterpage}
\usepackage{flushend}
\usepackage{titletoc}
\usepackage[toc,page]{appendix}
\usepackage{parskip}
\usepackage{graphicx} % Used by spritegrid and others.
\usepackage{tikzpagenodes}
\usepackage{imakeidx}
\usepackage[pagestyles,raggedright]{titlesec}
\usepackage[all]{nowidow}
\usepackage{hyperref}
\usepackage{aeb-minitoc}
\usepackage{fix-cm}
\usepackage{textpos}
\usepackage{enumitem}
% For multi-page tables in register maps

% For fixed-width columns in register maps
\usepackage{array}

% Makes tables with double-ruled lines look better
\usepackage{hhline}

% Makes better use of space for reference tables in appendix
\usepackage{multicol}

% Shaded tables with alternate rows colored for better legibility
% Best used with larger tables rather than small tables
\usepackage{colortbl}

\newcolumntype{L}[1]{>{\raggedright\let\newline\\\arraybackslash\hspace{0pt}}m{#1}}
\newcolumntype{C}[1]{>{\centering\let\newline\\\arraybackslash\hspace{0pt}}m{#1}}
\newcolumntype{R}[1]{>{\raggedleft\let\newline\\\arraybackslash\hspace{0pt}}m{#1}}

% For displaying Letter keys and the MEGA key
\input{elements/keys}

% For displaying print versions petscii character symbols
\input{elements/graphicsymbol}

% For Mega65 display of code, listings and screen activity
\input{elements/screenoutput}

% For displaying sprite data in a grid
\input{elements/spritegrid}

% Don't number sections
\setcounter{secnumdepth}{0}

\renewcommand{\indexname}{INDEX}
\renewcommand{\appendixtocname}{APPENDICES}
\renewcommand{\appendixpagename}{APPENDICES}
\renewcommand{\appendixpage}{%
  \clearpage\thispagestyle{empty}
    \pagecolor{blue}
     \begin{center}
       {
         \large
         % Put a nice amount of vertical space before the title
         \vspace*{2cm}
               {\large\Huge\textcolor{white}{\bf{APPENDICES}}}\\
             \vspace{\fill}
       }
     \end{center}
     \newpage\pagecolor{white}\clearpage
}

\makeatletter\chardef\pdf@shellescape=\@ne\makeatother

\setcounter{tocdepth}{5}

% 3.0 cm is the distance from left of page to bullet point.
% 2.8 cm is a fudge-factor to make multi-line section names be correctly lined up.
% \@B{〈length〉} is the amount to indent prior to〈sec-num >
% \@F{〈fmt〉} is the formatting for the title heading
% \@P{〈fmt〉} is the formatting for the page number (〈pg-num〉).

\TOCLevels{chapter}{section}
\begin{minitocfmt}{\chapmtoc}
\declaretocfmt{section}{\@F{\color{white}\hspace{3.0cm}\textbullet\hspace{0.25cm}\Large\bfseries}\@B{2.8cm}\@P{\mtocgobble}}
\declaretocfmt{section*}{\@F{\color{white}\hspace{3.0cm}\textbullet\hspace{0.25cm}\Large\bfseries}\@B{2.8cm}\@P{\mtocgobble}}
\end{minitocfmt}

% The following doesn't work,
% but the idea was to have a list of chapters on the part title pages
%\TOCLevels{part}{chapter}
%\begin{minitocfmt}{\partmtoc}
%\declaretocfmt{chapter}{\@F{\Large\bfseries\hspace{5cm}\color{white}\textbullet\hspace{1em}}\@B{1.5em}\@P{\mtocgobble}}
%\declaretocfmt{chapter*}{\@F{\Large\bfseries\hspace{5cm}\color{white}\textbullet\hspace{1em}}\@B{1.5em}\@P{\mtocgobble}}
%\declaretocfmt{section}{}
%\declaretocfmt{section*}{}
%\end{minitocfmt}

%\makeatletter
%\renewcommand*{\l@section}[1]{%
%  \@dottedtocline{1}{1.0em}{1em}{%
%    \numberline{\textbullet}%
%    % \numberline{$\cdot$}%
%    \gobble@numberline
%    #1%
%  }%
%}
%\makeatother

\input{fonts}

% Set margins for inner and outer pages in A5 book format
\usepackage[a5paper,nomarginpar,includemp,bottom=2cm,top=1cm,inner=1cm,outer=0.5cm, footskip = 1cm]{geometry}

% Some Computer Society conferences also require the compsoc mode option,
% but others use the standard conference format.
%
% If IEEEtran.cls has not been installed into the LaTeX system files,
% manually specify the path to it like:
% \documentclass[conference]{../sty/IEEEtran}

%% \input{setup}

% correct bad hyphenation here
\hyphenation{op-tical net-works semi-conduc-tor}

\makeindex[intoc]

\pagestyle{empty}

\begin{document}

% relax word wrapping with sloppy
\sloppy
% reduce overfull \hbox warnings
\hfuzz=5pt

%
% paper title
% Titles are generally capitalized except for words such as a, an, and, as,
% at, but, by, for, in, nor, of, on, or, the, to and up, which are usually
% not capitalized unless they are the first or last word of the title.
% Linebreaks \\ can be used within to get better formatting as desired.
% Do not put math or special symbols in the title.

\pagenumbering{roman}

\newpagestyle{onlynumber}{\setfoot[][{\bf\small\thepage}][]
                                  {} {\bf\small\thepage} {}}
\pagestyle{onlynumber}
\pagecolor{white}

%% XXX - big numbers are not in bold, because latex gets confused
\newcommand*{\justifyheading}{\raggedleft}
\definecolor{headingblue}{rgb}{0.5,0.5,1}

% \titleformat{command}[shape]
%   {format}
%   {label}
%   {sep}
%   {before}
%   [after]

% ***************
% PART title page
% ***************

\titleclass{\part}{top}
\titleformat{\part}[display]
   {\thispagestyle{empty}\pagecolor{blue}\normalfont\huge\bfseries\justifyheading}
   {\textcolor{white}{\fontsize{50}{65}\selectfont\bf{PART}\quad{\fontsize{100}{130}\selectfont \bf{\serifed\thepart}}}}
   {20pt}
   {\Huge\textcolor{white}}
   [\newpage\pagecolor{white}\textcolor{black}]

% ******************
% CHAPTER title page
% ******************

\titleformat{\chapter}[display]
   {\thispagestyle{empty}\pagecolor{blue}\normalfont\huge\bfseries\justifyheading}
   {\textcolor{white}{\MakeUppercase{\chaptertitlename}\quad{\fontsize{100}{130}\selectfont \bf\thechapter}}}
   {20pt}
   {\Huge\textcolor{white}}
   [{\chapmtoc\insertminitoc}\newpage\pagecolor{white}\textcolor{black}\cleardoublepage]

% ******************
% SECTION title page
% ******************
   
\titleformat{\section}[display]
   {\raggedright}
   {\thesection}
   {20pt}
   {\huge\bf\color{headingblue}\uppercase}
   [\color{black}]

\titlecontents{\chapter}

\NewDocumentCommand{\thesistitle} { o m }{%
 \IfValueTF{#1}{\def\shorttitle{#1}}{\def\shorttitle{#2}}%
 \def\@title{#2}%
 \def\ttitle{#2}%
}
\DeclareDocumentCommand{\author}{m}{\newcommand{\authorname}{#1}\renewcommand{\@author}{#1}}
\NewDocumentCommand{\supervisor}{m}{\newcommand{\supname}{#1}}
\NewDocumentCommand{\examiner}{m}{\newcommand{\examname}{#1}}
\NewDocumentCommand{\degree}{m}{\newcommand{\degreename}{#1}}
\NewDocumentCommand{\addresses}{m}{\newcommand{\addressname}{#1}}
\NewDocumentCommand{\university}{m}{\newcommand{\univname}{#1}}
\NewDocumentCommand{\department}{m}{\newcommand{\deptname}{#1}}
\NewDocumentCommand{\group}{m}{\newcommand{\groupname}{#1}}
\NewDocumentCommand{\faculty}{m}{\newcommand{\facname}{#1}}
\NewDocumentCommand{\subject}{m}{\newcommand{\subjectname}{#1}}
\NewDocumentCommand{\keywords}{m}{\newcommand{\keywordnames}{#1}}

\newcommand{\checktoopen}{% New command to move content to the next page which prints to the next odd page if twosided mode is active
        \if@openright\cleardoublepage\else\clearpage\fi
        \ifdef{\phantomsection}{\phantomsection}{}% The \phantomsection command is necessary for hyperref to jump to the correct page
}



%----------------------------------------------------------------------------------------
%	THESIS INFORMATION
%----------------------------------------------------------------------------------------

\thesistitle{Verb Tenses and Time-Travel: Getting Past the Future Semiconditionally Modified Subinverted Plagal Past Subjunctive Intentional} % Your thesis title, this is used in the title and abstract, print it elsewhere with \ttitle
\supervisor{Dr. Paul \textsc{Gardner-Stephen}} % Your supervisor's name, this is used in the title page, print it elsewhere with \supname
\examiner{} % Your examiner's name, this is not currently used anywhere in the template, print it elsewhere with \examname
\degree{Bachelour of Engineering(Electronics)(Honours)} % Your degree name, this is used in the title page and abstract, print it elsewhere with \degreename
\author{Dan Streetmentioner} % Your name, this is used in the title page and abstract, print it elsewhere with \authorname
\addresses{} % Your address, this is not currently used anywhere in the template, print it elsewhere with \addressname

\subject{Electronic Engineering} % Your subject area, this is not currently used anywhere in the template, print it elsewhere with \subjectname
\keywords{} % Keywords for your thesis, this is not currently used anywhere in the template, print it elsewhere with \keywordnames
\university{\href{https://www.flinders.edu.au/}{Flinders University}} % Your university's name and URL, this is used in the title page and abstract, print it elsewhere with \univname
\department{\href{https://www.flinders.edu.au/college-science-engineering}{College of Science and Engineering}} % Your department's name and URL, this is used in the title page and abstract, print it elsewhere with \deptname
\group{The Douglas Adams Institute For Implausible Linguistics} % Your research group's name and URL, this is used in the title page, print it elsewhere with \groupname
\faculty{\href{}{}} % Your faculty's name and URL, this is used in the title page and abstract, print it elsewhere with \facname

\AtBeginDocument{
\hypersetup{pdftitle=\ttitle} % Set the PDF's title to your title
\hypersetup{pdfauthor=\authorname} % Set the PDF's author to your name
\hypersetup{pdfkeywords=\keywordnames} % Set the PDF's keywords to your keywords
}

\begin{document}

\frontmatter % Use roman page numbering style (i, ii, iii, iv...) for the pre-content pages

\pagestyle{plain} % Default to the plain heading style until the thesis style is called for the body content

%----------------------------------------------------------------------------------------
%	TITLE PAGE
%----------------------------------------------------------------------------------------

\begin{titlepage}
\begin{center}

\vspace*{.06\textheight}
{\scshape\LARGE \univname\par}\vspace{1.5cm} % University name
\textsc{\Large Honours Thesis}\\[0.5cm] % Thesis type

\HRule \\[0.4cm] % Horizontal line
{\huge \bfseries \ttitle\par}\vspace{0.4cm} % Thesis title
\HRule \\[1.5cm] % Horizontal line
 
\begin{minipage}[t]{0.4\textwidth}
\begin{flushleft} \large
\emph{Author:}\\
{\authorname} % Author name - remove the \href bracket to remove the link
\end{flushleft}
\end{minipage}
\begin{minipage}[t]{0.4\textwidth}
\begin{flushright} \large
\emph{Supervisor:} \\
{\supname} % Supervisor name - remove the \href bracket to remove the link  
\end{flushright}
\end{minipage}\\[3cm]
 
\vfill

\large \textit{A thesis submitted in fulfilment of the requirements\\ for the degree of \degreename}\\[0.3cm] % University requirement text
%%\groupname\\\deptname\\[2cm] % Research group name and department name
 
\vfill

{\large \today}\\[4cm] % Date
%\includegraphics{Logo} % University/department logo - uncomment to place it
 
\vfill
\end{center}
\end{titlepage}

%----------------------------------------------------------------------------------------
%	DECLARATION PAGE
%----------------------------------------------------------------------------------------

\begin{declaration}
\addchaptertocentry{\authorshipname} % Add the declaration to the table of contents
\noindent I, \authorname, declare that this thesis titled, \enquote{\ttitle} and the work presented in it are my own. I confirm that:

\begin{itemize} 
\item This work was done wholly while in candidature for a degree of \degreename.
\item This document is in accordance with the plagiarism policy of \univname.
\item Where any part of this thesis has previously been submitted for a degree or any other qualification at this University or any other institution, this has been clearly stated.
\item Where I have consulted the published work of others, this is always clearly attributed.
\item Where I have quoted from the work of others, the source is always given. With the exception of such quotations, this thesis is entirely my own work.
\item I have acknowledged all main sources of help.
\item Where the thesis is based on work done by myself jointly with others, I have made clear exactly what was done by others and what I have contributed myself.\\
\end{itemize}
 
\noindent Signed:\\
\rule[0.5em]{25em}{0.5pt} % This prints a line for the signature
 
\noindent Date:\\
\rule[0.5em]{25em}{0.5pt} % This prints a line to write the date
\end{declaration}

\cleardoublepage

%----------------------------------------------------------------------------------------
%	QUOTATION PAGE
%----------------------------------------------------------------------------------------

\vspace*{0.2\textheight}

\noindent\enquote{\itshape One of the major problems encountered in time travel is not that of becoming your own father or mother. There is no problem in becoming your own father or mother that a broad-minded and well-adjusted family can't cope with. There is no problem with changing the course of history—the course of history does not change because it all fits together like a jigsaw. All the important changes have happened before the things they were supposed to change and it all sorts itself out in the end.

The major problem is simply one of grammar, and the main work to consult in this matter is Dr. Dan Streetmentioner's Time Traveler's Handbook of 1001 Tense Formations. It will tell you, for instance, how to describe something that was about to happen to you in the past before you avoided it by time-jumping forward two days in order to avoid it. The event will be descibed differently according to whether you are talking about it from the standpoint of your own natural time, from a time in the further future, or a time in the further past and is futher complicated by the possibility of conducting conversations while you are actually traveling from one time to another with the intention of becoming your own mother or father.

Most readers get as far as the Future Semiconditionally Modified Subinverted Plagal Past Subjunctive Intentional before giving up; and in fact in later aditions of the book all pages beyond this point have been left blank to save on printing costs.

The Hitchhiker's Guide to the Galaxy skips lightly over this tangle of academic abstraction, pausing only to note that the term "Future Perfect" has been abandoned since it was discovered not to be.}\bigbreak

\hfill The Hitch Hiker's Guide To The Galaxy

%----------------------------------------------------------------------------------------
%	ABSTRACT PAGE
%----------------------------------------------------------------------------------------

\begin{abstract}
  \addchaptertocentry{\abstractname} % Add the abstract to the table of contents

One of the major problems encountered in time travel is not that of becoming your own father or mother. There is no problem in becoming your own father or mother that a broad-minded and well-adjusted family can't cope with. There is no problem with changing the course of history—the course of history does not change because it all fits together like a jigsaw. All the important changes have happened before the things they were supposed to change and it all sorts itself out in the end.

The major problem is simply one of grammar, for instance, how to describe something that was about to happen to you in the past before you avoided it by time-jumping forward two days in order to avoid it. The event will be descibed differently according to whether you are talking about it from the standpoint of your own natural time, from a time in the further future, or a time in the further past and is futher complicated by the possibility of conducting conversations while you are actually traveling from one time to another with the intention of becoming your own mother or father.

\end{abstract}

%----------------------------------------------------------------------------------------
%	ACKNOWLEDGEMENTS
%----------------------------------------------------------------------------------------

\begin{acknowledgements}
\addchaptertocentry{\acknowledgementname} % Add the acknowledgements to the table of contents
I wish to thank the editorial staff of Megadodo Publications (Ursa Minor Beta) for their support throughout this thesis,
as well as Clearance Textiles Limited of Strood for providing the numerous bath towels consumed as part of this thesis.
\end{acknowledgements}

%----------------------------------------------------------------------------------------
%	LIST OF CONTENTS/FIGURES/TABLES PAGES
%----------------------------------------------------------------------------------------

\tableofcontents % Prints the main table of contents

\listoffigures % Prints the list of figures

%\listoftables % Prints the list of tables

%----------------------------------------------------------------------------------------
%	ABBREVIATIONS
%----------------------------------------------------------------------------------------

%\begin{abbreviations}{ll} % Include a list of abbreviations (a table of two columns)

%\textbf{LAH} & \textbf{L}ist \textbf{A}bbreviations \textbf{H}ere\\
%\textbf{WSF} & \textbf{W}hat (it) \textbf{S}tands \textbf{F}or\\

%\end{abbreviations}

%----------------------------------------------------------------------------------------
%	PHYSICAL CONSTANTS/OTHER DEFINITIONS
%----------------------------------------------------------------------------------------

%\begin{constants}{lr@{${}={}$}l} % The list of physical constants is a three column table

% The \SI{}{} command is provided by the siunitx package, see its documentation for instructions on how to use it

%Speed of Light & $c_{0}$ & \SI{2.99792458e8}{\meter\per\second} (exact)\\
%Constant Name & $Symbol$ & $Constant Value$ with units\\

%\end{constants}

%----------------------------------------------------------------------------------------
%	SYMBOLS
%----------------------------------------------------------------------------------------

%\begin{symbols}{lll} % Include a list of Symbols (a three column table)

%$a$ & distance & \si{\meter} \\
%$P$ & power & \si{\watt} (\si{\joule\per\second}) \\
%Symbol & Name & Unit \\

%\addlinespace % Gap to separate the Roman symbols from the Greek

%$\omega$ & angular frequency & \si{\radian} \\

%\end{symbols}

%----------------------------------------------------------------------------------------
%	DEDICATION
%----------------------------------------------------------------------------------------

%\dedicatory{For/Dedicated to/To my\ldots} 

%----------------------------------------------------------------------------------------
%	THESIS CONTENT - CHAPTERS
%----------------------------------------------------------------------------------------

\mainmatter % Begin numeric (1,2,3...) page numbering

\pagestyle{thesis} % Return the page headers back to the "thesis" style

\include{Chapters/Chapter1}
\include{Chapters/Chapter2}
\include{Chapters/Chapter3}
\include{Chapters/Chapter4}
\include{Chapters/Chapter5}


\nocite{*}
\bibliographystyle{IEEEtran}
\bibliography{references}

\end{document}
